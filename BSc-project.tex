\documentclass[usenames,dvipsnames]{beamer}

\usetheme{Madrid}
\usecolortheme{beaver}

\usepackage{tikz}
\usetikzlibrary{shapes.geometric, arrows, positioning}
\usepackage[absolute,overlay]{textpos}
\usepackage{mathtools}
\usepackage{verbatim}
\usepackage[absolute,overlay]{textpos}

\title[BSc project]{BSc project: detection and visualization of bird movement recorded by weather radar}
\author[L. Verlinden, Z. Koma, W. Bouten]{Liesbeth Verlinden, Zs\'ofi Koma, dr. ir. prof. Willem Bouten}
\date{\today}

\newenvironment{myblock}[3]{%
	\setbeamercolor{block body}{#2}
	\setbeamercolor{block title}{#3}
\begin{block}{#1}}{\end{block}}


\begin{document}

%title page
\begin{frame}
	\titlepage
\end{frame}


%toc
\begin{frame}{Table of content}
	\begin{enumerate}
		\item Radar basics
		\begin{itemize}
			\item [-] storage of radar data
		\end{itemize}
		\item Bird detection by radar
		\item 3D point cloud
		\item Aim of the project
	\end{enumerate}
\end{frame}


%radar basics: principle
\begin{frame}{Radar basics: the principle}
\begin{center}
	\includegraphics[height=1cm]{./pics/radar-pulse-1.png}
	\hspace{1cm}
	\includegraphics[height=1cm]{./pics/radar-pulse-4.png}
	\hspace{1cm}
	\includegraphics[height=1cm]{./pics/radar-pulse-6.png}
\end{center}

\begin{itemize}
	\item Electromagnetic pulse %(NL: C-band) %wavelength: 5.3cm, freq range 4-8 GHz
	\item Echoes: 
	\begin{itemize}
		\item[-] returned power/reflectivity
		\item[-] radial/Doppler velocity, calculated from the shift in wavelength
	\end{itemize}
\end{itemize}
\end{frame}


%radar basics: the beam
\begin{frame}{Radar basics: the beam}
\begin{itemize}
	\item Transmitter - receiver
	\item Range gate/bin
	\begin{center}
		\includegraphics[height=1.5cm]{./pics/range-gate}
		\vspace{0.5cm}
		\includegraphics[height=2.5cm]{./pics/range-gate2}
	\end{center}
	\item N$\,^{\circ}$ of range gates depends on wavelength
\end{itemize}	
\end{frame}


%radar basics: range gate/bin
\begin{frame}{Radar basics: range gate/bin}
Each range gate has:
\begin{itemize}
	\item certain amount of returned power/reflectivity
	\item radial/Doppler velocity
\end{itemize}
\end{frame}


%radar basics: scans and volumes
\begin{frame}{Radar basics: scans and volumes}
\centering
\begin{tikzpicture}
\node (ela1) [] {\includegraphics[height=2.5cm]{./pics/ela1}};
\node (ela2) [right=0.5cm of ela1]  {\includegraphics[height=2.5cm]{./pics/ela2}};

\draw[thick, dashed] (-1.6,0) -- (1.5, 0.6);
\draw[thick, dashed] (4,0.1) -- (6.7, 1);

\draw[->, thick, blue] (1.2,-1) arc (0:27:3cm);
\draw[->, thick, blue] (6.7,-1) arc (0:35:3cm);

\node[blue, below=0.1cm of ela1] {elevation angle 0.5$\,^{\circ}$};
\node[blue, below=0.1cm of ela2] {elevation angle 1.1$\,^{\circ}$};

\draw[red, ->, thick] (-0.6,1)  arc (0:300:1cm and 0.4cm);
\draw[red, ->, thick] (5,1)  arc (0:300:1cm and 0.4cm);
 
\node[below right = 1cm and -2.5cm of ela1] (3rad) {\includegraphics[height=3cm]{./pics/3d-radar.jpg}};
\end{tikzpicture}
\end{frame}



%radar basics: the beam across distance
\begin{frame}{Radar basics: the beam across distance}
	\begin{center}	
		\includegraphics[height=4.4cm]{./pics/beam-width}
		\hspace{.25cm}
		\includegraphics[height=4.4cm]{./pics/beam-height}
	\end{center}
\end{frame}




%radar basics: doppler spectrum
\begin{frame}{Radar basics: Doppler spectrum}
	Doppler spectrum for one range gate:
	\vspace{0.25cm}
	\begin{center}
		\includegraphics[height=3cm]{./pics/ground-clutter1}
	\end{center}	
	
\end{frame}

%radar basics: clutter filter
\begin{frame}{Radar basics: radar processor clutter filtering}
	\begin{center}
		\includegraphics[height=3cm]{./pics/ground-clutter2}
	\end{center}

	\begin{itemize}				
		\item Doppler filter, during data collection
		\item Removes the peak in returned power around radial velocity = 0 m/s via a sharp high-pass filter and reconstructs the data
	\end{itemize}
\end{frame}



\begin{frame}{Radar basics: the PPI}
\centering
\begin{tikzpicture}
\node[inner sep=0pt]  {\includegraphics[height=5cm]{./pics/ppi}};
\node at (5, 0.2) {\includegraphics[height=3cm]{./pics/fragm-ppi}};

\draw[thick] (-1,-0.5) -- (0,-0.5) -- (0,0.5) -- (-1,0.5) -- (-1,-0.5);	
\draw[thick] (0,0.5) -- (3.25,1.69);
\draw[thick] (0,-0.5) -- (3.25,-1.29);

\draw[thick] (5.85,-0.5) circle (0.25cm);
\draw[thick, ->] (6,-2) --(5.85,-0.75);
\node at (6, -2.3) {range gate};
\end{tikzpicture}
\end{frame}


%data storage
\begin{frame}{Storage radar data: hdf5}
See example
\end{frame}


%bird detection: meteo vs non-meteo scatter
\begin{frame}{Bird detection: meteo vs non-meteo scatter}
\begin{itemize}
	\item 1940$^s$, ``angles''
\end{itemize}
\begin{center}
		\includegraphics[height=7cm]{./pics/case-studies.jpg}
\end{center}
\end{frame}


%bird detection: vol2bird
\begin{frame}{Bird detection: vol2bird}
\begin{itemize}
	\item Detection algorithm \textit{vol2bird} 
	\begin{itemize}
		\item[-] reflectivity and variability in radial velocity
		\item[-] averages per 200m and between 5-25km $\Rightarrow$ 2D image
	\end{itemize}
\end{itemize}
		\begin{center}
	\begin{tikzpicture}
	\node[inner sep=0pt] {\includegraphics[height=5cm]{./pics/output-layer}};
	\node at (5, 0.2) {\includegraphics[height=3cm]{./pics/vpb-part}};
	
	\draw[thick] (-1,-0.5) -- (0,-0.5) -- (0,0.5) -- (-1,0.5) -- (-1,-0.5);	
	\draw[thick, ->] (0,0) --(5,1.3);			
	\draw[thick] (5.25,1.35) circle (0.25cm);
	\end{tikzpicture}
\end{center}
\end{frame}
	
	
%3D point cloud	
\begin{frame}{3D point cloud}
\end{frame}	
	

%aim of the project
\begin{frame}{Aim of the project}
The aim of the project is to make the separation between biological (bird) and rain echoes using machine learning techniques. The filtered data should then be plotted in a 3D point cloud, coloured according to a specified scale.
\end{frame}	
	
\end{document}